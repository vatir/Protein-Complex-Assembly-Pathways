% Template for Biophysics paper in LaTeX
%
% To compile into a document, run
% latex biophys_latex_template
% bibtex biophys_latex_template (if bib file and bst file is included in TeX file)
% latex biophys_latex_template (run 2-3 times repeatedly)
% dvips biophys_latex_template.dvi
%
% or replace the latex command by the pdflatex command in the lines above to
% generate a PDF file and use acroread or xpdf for viewing and
% printing instead of the postscript generating program dvips

% Use standard biophys document class with default font size
% and typeset in one column. If you need to typeset in two column
% then give the option "twocolumn" ie \documentclass[twocolumn]{biophys}
\documentclass{biophys}
\usepackage{helvet,times}
\usepackage{bm,textcomp}

%\jno{kxl014} %journal number
\gridframe{N}%option for grid around the text "Y" or "N"
\cropmark{N}%option for cropmark around the text "Y" or "N"

%The first page number and last page number automatically generated.
%To change the page number \setcounter{page}{10} automatically reset
%the first and last page number but two times compilation required.
%If you want to edit the page range in catch line
% then edit the below two lines
%\fpage{}
%\lpage{}
%For update volume number, activate below command
%\volume{00}
%For update issue number, activate below command
%\issue{00}
%For update Month, activate below command
%\Month{Month}
%For update Year, activate below command
%\Year{Year}


% Packages to load (all standard on a modern LaTeX system on Linux)

% Make doublespaced ugly typography required for mysterious
% reasons by most journals - comment out for normal output
%\usepackage{setspace}
%\doublespacing
% AMS-Math package to have nice multi-line equations and other goodies
\usepackage{amsmath}
\usepackage{physics}
% Show labels for easy orientation, comment out for final version
% \usepackage{showlabels}

% EPS/PDF graphics
% Place figures in the document directory in both the EPS and PDF
% formats, e.g., fig_1.eps and fig_1.pdf. Use the includegraphics
% command without file extension, e.g. \includegraphics*[width=3.25in]{fig_1}
% The pdflatex or latex programs then work automagically with the
% appropriate formats.  EPS figures can be converted to PDF using
% the epstopdf program present on most Linux disributions. Epstopdf and graphicx
% are included in biophys class file.
% \usepackage{graphicx}

% Citation style in the text: numbers in parenthesis, sorted by their
% order in the list of references.
% Uses a range if possible: (1-3), not (1,2,3)

\usepackage[round,numbers,sort&compress]{natbib}

% Bibliography style (requires the style file biophysj.bst in the
% document directory)

%\bibliographystyle{biophysj}

% Numbering style in the list of references: a number followed by a period


% Examples of special definitions (amsmath package required)
%\newcommand{\erf}{\operatorname{erf}}        % error function
%\newcommand{\erfc}{\operatorname{erfc}}      % complementary error function
%\newcommand{\BibTeX}{\textsc{Bib}\TeX}       % corect BibTeX appearance

%\renewcommand{\baselinestretch}{1.5}
\linespread{1.5}

\newcommand{\etal}{\textit{et al}.\ }
\newcommand{\ie}{\textit{i}.\textit{e}.,\ }
\newcommand{\eg}{\textit{e}.\textit{g}.,\ }
\newcommand{\apriori}{\textit{a priori}\xspace}
\newcommand{\Apriori}{\textit{A priori}\xspace}

\providecommand{\e}[1]{\ensuremath{\times 10^{#1}}\xspace}
\providecommand{\orderm}[1]{\ensuremath{10^{#1}}\xspace}

\newcommand{\kon}{\ensuremath{k_{\textrm{on}}}\xspace}
\newcommand{\koff}{\ensuremath{k_{\textrm{off}}}\xspace}

\newcommand{\konunit}{\ensuremath{\textrm{M}^{-1}\textrm{s}^{-1}}\xspace}
\newcommand{\koffunit}{\ensuremath{\textrm{s}^{-1}}\xspace}
\newcommand{\synunit}{\ensuremath{\textrm{M}^{-1}\textrm{s}^{-1}}\xspace}
\newcommand{\delunit}{\ensuremath{\textrm{s}^{-1}}\xspace}

\newcommand{\M}{\ensuremath{\textrm{M}}\xspace}
\newcommand{\nM}{\ensuremath{\textrm{n\M}}\xspace}
\newcommand{\Concunit}{\M}
\newcommand{\kdunit}{\Concunit}


\newcommand{\species}[1]{\ensuremath{\textrm{S}_{#1}}\xspace}


%\newcommand{\konAA}{\ensuremath{k_{\textrm{on,A\!A}}}}
%\newcommand{\koffAA}{\ensuremath{k_{\textrm{off,A\!A}}}}

\newcommand{\Entity}[1]{\ensuremath{\textrm{#1}}\xspace}
\newcommand{\Conc}[1]{\ensuremath{\textrm{[#1{}]}}\xspace}
\newcommand{\Total}[1]{\ensuremath{\quad\mathclap{\Conc{#1}_\textrm{Total}}\qquad}\xspace}
\newcommand{\TotalNoSpacing}[1]{\ensuremath{\Conc{#1}_\textrm{Total}}}

\newcommand{\Initial}[1]{\ensuremath{\Conc{#1}_\textrm{0}}\xspace}
\newcommand{\InitialSq}[1]{\ensuremath{\Conc{#1}_\textrm{0}^2}\xspace}

\newcommand{\TotalSS}[1]{\ensuremath{\quad\mathclap{\Conc{#1}_\textrm{Total,SS}}\qquad}\xspace}
\newcommand{\TotalSSNoSpacing}[1]{\ensuremath{\Conc{#1}_\textrm{Total,SS}}\xspace}
\newcommand{\ConcSS}[1]{\ensuremath{\Conc{#1}_\textrm{SS}}\xspace}

%\newcommand{\DCDt}[1]{\dfrac{\mathrm{d}\Conc{#1}}{\mathrm{d}t}}
\newcommand{\DCDtNoSpacing}[1]{\ensuremath{{\dfrac{\mathrm{d}\Conc{#1{}}}{\mathrm{d}t}}}}

%\newcommand{\DCDtNoSpacing}[1]{\ensuremath{\mathclap{\dfrac{\mathrm{d}\Conc{#1}}{\mathrm{d}t}}}}
\newcommand{\DCDt}[1]{\ensuremath{\quad\mathclap{\DCDtNoSpacing{#1}}\qquad}}

\newcommand{\A}{\ensuremath{\Entity{A}}\xspace}
\newcommand{\B}{\ensuremath{\Entity{B}}\xspace}
\newcommand{\AB}{\ensuremath{\Entity{AB}}\xspace}
\newcommand{\BB}{\ensuremath{\Entity{BB}}\xspace}
\newcommand{\ABB}{\ensuremath{\Entity{ABB}}\xspace}
\newcommand{\AtoA}{\ensuremath{\Entity{AA}}\xspace}
\renewcommand{\AA}{\AtoA}

\renewcommand{\P}{\ensuremath{\Entity{P}}\xspace}
\newcommand{\N}{\ensuremath{\Entity{N}}\xspace}
\newcommand{\PN}{\ensuremath{\Entity{PN}}\xspace}
\newcommand{\PPN}{\ensuremath{\Entity{PPN}}\xspace}

\newcommand{\invivo}{\emph{in vivo}-like\xspace}
\newcommand{\invitro}{\emph{in vitro}-like\xspace}

\newcommand{\del}{\ensuremath{\delta{}}\xspace}
\newcommand{\syn}{\ensuremath{Q{}}\xspace}

\newcommand{\konX}[1]{\ensuremath{k_{\textrm{on},\textrm{#1}}}\xspace}
\newcommand{\koffX}[1]{\ensuremath{k_{\textrm{off},\textrm{#1}}}\xspace}
\newcommand{\kd}{\ensuremath{K_d}\xspace}
\newcommand{\kdSub}[1]{\ensuremath{K_{d,\textrm{#1}}}\xspace}
\newcommand{\kdWSpacing}{\ensuremath{\quad\mathclap{\kd}\quad}}
\newcommand{\kdSubWSpacing}[1]{\ensuremath{\quad\mathclap{\kdSub{#1}}\quad}}
\newcommand{\keff}{\ensuremath{k^\textrm{eff}}\xspace}

\newcommand{\konAB}{\konX{\AB}}
\newcommand{\koffAB}{\koffX{\AB}}

\newcommand{\konAA}{\konX{\AA}}
\newcommand{\koffAA}{\koffX{\AA}}

\newcommand{\konPN}{\konX{\PN}}
\newcommand{\koffPN}{\koffX{\PN}}
\newcommand{\konPPN}{\konX{\PPN}}
\newcommand{\koffPPN}{\koffX{\PPN}}
\newcommand{\kdPN}{\kdSub{\PN}}
\newcommand{\kdPPN}{\kdSub{\PPN}}


\begin{document}
\setlength\parindent{0pt}
%\LARGE
\large
%\begin{align*}
%\begin{align*}
%\begin{equation}
%\begin{split}
\begin{equation*}k^{eff}_{i,j} = \alpha \kon \kdSub{1}^i \kdSub{2}^j \mathrm{e}^{(i+j-1)\Delta G_p^0/RT}\end{equation*}
\begin{equation*}\alpha = c_0^{-i - j}\end{equation*}
\begin{equation*}k_p = \orderm{6}\ \ \konunit\end{equation*}
\begin{align*}\DCDtNoSpacing{\species{1}} &= \kon{}(-6\Conc{\species{1}}^2-4\Conc{\species{1}}\Conc{\species{2}}-3\Conc{\species{1}}\Conc{\species{3}}-3\Conc{\species{1}}\Conc{\species{4}}-2\Conc{\species{1}}\Conc{\species{5}}-3\Conc{\species{1}}\Conc{\species{6}}-2\Conc{\species{1}}\Conc{\species{7}}\\
 & \quad -2\Conc{\species{1}}\Conc{\species{8}}-3\Conc{\species{1}}\Conc{\species{9}}-2\Conc{\species{1}}\Conc{\species{10}}-\Conc{\species{1}}\Conc{\species{11}})+2\Conc{\species{2}}k^{eff}_{0,1}\\
 & \quad +2\Conc{\species{3}}k^{eff}_{1,0}+\Conc{\species{4}}k^{eff}_{1,0}+\Conc{\species{4}}k^{eff}_{0,1}+2\Conc{\species{5}}k^{eff}_{1,0}+\Conc{\species{6}}k^{eff}_{1,0}+\Conc{\species{6}}k^{eff}_{0,1}+2\Conc{\species{7}}k^{eff}_{1,0}\\
 & \quad +4\Conc{\species{8}}k^{eff}_{1,1}+3\Conc{\species{9}}k^{eff}_{2,0}+2\Conc{\species{10}}k^{eff}_{2,0}+\Conc{\species{10}}k^{eff}_{0,1}+2\Conc{\species{11}}k^{eff}_{2,1}+\Conc{\species{11}}k^{eff}_{2,0}+2\Conc{\species{11}}k^{eff}_{1,1}+6\Conc{\species{12}}k^{eff}_{2,1}-\delta\Conc{\species{1}}+Q\\[0.5em]
\DCDtNoSpacing{\species{2}} &= \kon{}(\Conc{\species{1}}^2-4\Conc{\species{1}}\Conc{\species{2}}-8\Conc{\species{2}}^2-2\Conc{\species{2}}\Conc{\species{3}}-2\Conc{\species{2}}\Conc{\species{4}}-2\Conc{\species{2}}\Conc{\species{6}}-2\Conc{\species{2}}\Conc{\species{8}}) \\
 & \quad -\Conc{\species{2}}k^{eff}_{0,1}+\Conc{\species{4}}k^{eff}_{1,0}+\Conc{\species{6}}k^{eff}_{1,0}+2\Conc{\species{8}}k^{eff}_{2,0}+\Conc{\species{10}}k^{eff}_{2,0}+2\Conc{\species{11}}k^{eff}_{3,0}\\
 & \quad +3\Conc{\species{12}}k^{eff}_{4,0}-\delta\Conc{\species{2}}\\[0.5em]
\DCDtNoSpacing{\species{3}} &= \kon{}(2\Conc{\species{1}}^2-3\Conc{\species{1}}\Conc{\species{3}}-2\Conc{\species{2}}\Conc{\species{3}}-6\Conc{\species{3}}^2-\Conc{\species{3}}\Conc{\species{4}}-\Conc{\species{3}}\Conc{\species{6}}-3\Conc{\species{3}}\Conc{\species{9}}\\
 & \quad -\Conc{\species{3}}\Conc{\species{10}})-\Conc{\species{3}}k^{eff}_{1,0}+\Conc{\species{4}}k^{eff}_{0,1}+2\Conc{\species{5}}k^{eff}_{0,1}+\Conc{\species{6}}k^{eff}_{0,1}+2\Conc{\species{7}}k^{eff}_{0,1}\\
 & \quad +2\Conc{\species{8}}k^{eff}_{0,2}+3\Conc{\species{9}}k^{eff}_{2,0}+\Conc{\species{10}}k^{eff}_{2,0}+2\Conc{\species{11}}k^{eff}_{2,1}+\Conc{\species{11}}k^{eff}_{0,2}+6\Conc{\species{12}}k^{eff}_{2,2}-\delta\Conc{\species{3}}\\[0.5em]
\DCDtNoSpacing{\species{4}} &= \kon{}(2\Conc{\species{1}}\Conc{\species{2}}+\Conc{\species{1}}\Conc{\species{3}}-3\Conc{\species{1}}\Conc{\species{4}}-2\Conc{\species{2}}\Conc{\species{4}}-\Conc{\species{3}}\Conc{\species{4}}-2\Conc{\species{4}}^2)-\Conc{\species{4}}k^{eff}_{1,0}\\
 & \quad -\Conc{\species{4}}k^{eff}_{0,1}+2\Conc{\species{5}}k^{eff}_{1,0}+2\Conc{\species{8}}k^{eff}_{1,1}+\Conc{\species{10}}k^{eff}_{2,0}+\Conc{\species{11}}k^{eff}_{3,0}+\Conc{\species{11}}k^{eff}_{2,1}\\
 & \quad +6\Conc{\species{12}}k^{eff}_{4,1}-\delta\Conc{\species{4}}\\[0.5em]
\DCDtNoSpacing{\species{5}} &= \kon{}(\Conc{\species{1}}\Conc{\species{4}}-2\Conc{\species{1}}\Conc{\species{5}}+\Conc{\species{3}}^2)-2\Conc{\species{5}}k^{eff}_{1,0}-\Conc{\species{5}}k^{eff}_{0,1}+\Conc{\species{11}}k^{eff}_{2,1}-\delta\Conc{\species{5}}\\[0.5em]
\DCDtNoSpacing{\species{6}} &= \kon{}(2\Conc{\species{1}}\Conc{\species{2}}+\Conc{\species{1}}\Conc{\species{3}}-3\Conc{\species{1}}\Conc{\species{6}}-2\Conc{\species{2}}\Conc{\species{6}}-\Conc{\species{3}}\Conc{\species{6}}-2\Conc{\species{6}}^2)-\Conc{\species{6}}k^{eff}_{1,0}\\
 & \quad -\Conc{\species{6}}k^{eff}_{0,1}+2\Conc{\species{7}}k^{eff}_{1,0}+2\Conc{\species{8}}k^{eff}_{1,1}+\Conc{\species{10}}k^{eff}_{2,0}+\Conc{\species{11}}k^{eff}_{3,0}+\Conc{\species{11}}k^{eff}_{2,1}\\
 & \quad +6\Conc{\species{12}}k^{eff}_{4,1}-\delta\Conc{\species{6}}\\[0.5em]
\DCDtNoSpacing{\species{7}} &= \kon{}(\Conc{\species{1}}\Conc{\species{6}}-2\Conc{\species{1}}\Conc{\species{7}}+\Conc{\species{3}}^2)-2\Conc{\species{7}}k^{eff}_{1,0}-\Conc{\species{7}}k^{eff}_{0,1}+\Conc{\species{11}}k^{eff}_{2,1}-\delta\Conc{\species{7}}\\[0.5em]
\DCDtNoSpacing{\species{8}} &= \kon{}(\Conc{\species{1}}\Conc{\species{4}}+\Conc{\species{1}}\Conc{\species{6}}-2\Conc{\species{1}}\Conc{\species{8}}+4\Conc{\species{2}}^2-2\Conc{\species{2}}\Conc{\species{8}}+\Conc{\species{3}}^2)-4\Conc{\species{8}}k^{eff}_{1,1}\\
 & \quad -\Conc{\species{8}}k^{eff}_{2,0}-\Conc{\species{8}}k^{eff}_{0,2}+\Conc{\species{11}}k^{eff}_{2,0}+3\Conc{\species{12}}k^{eff}_{4,0}-\delta\Conc{\species{8}}\\[0.5em]
\DCDtNoSpacing{\species{9}} &= \kon{}(\Conc{\species{1}}\Conc{\species{3}}-3\Conc{\species{1}}\Conc{\species{9}}-3\Conc{\species{3}}\Conc{\species{9}}-6\Conc{\species{9}}^2)-3\Conc{\species{9}}k^{eff}_{2,0}+\Conc{\species{10}}k^{eff}_{0,1}\\
 & \quad +\Conc{\species{11}}k^{eff}_{0,2}+2\Conc{\species{12}}k^{eff}_{0,3}-\delta\Conc{\species{9}}\\[0.5em]
\DCDtNoSpacing{\species{10}} &= \kon{}(\Conc{\species{1}}\Conc{\species{4}}+\Conc{\species{1}}\Conc{\species{6}}+3\Conc{\species{1}}\Conc{\species{9}}-2\Conc{\species{1}}\Conc{\species{10}}+2\Conc{\species{2}}\Conc{\species{3}}-\Conc{\species{3}}\Conc{\species{10}}) \\
 & \quad -3\Conc{\species{10}}k^{eff}_{2,0}-\Conc{\species{10}}k^{eff}_{0,1}+2\Conc{\species{11}}k^{eff}_{1,1}+6\Conc{\species{12}}k^{eff}_{2,2}-\delta\Conc{\species{10}}\\[0.5em]
\DCDtNoSpacing{\species{11}} &= \kon{}(2\Conc{\species{1}}\Conc{\species{5}}+2\Conc{\species{1}}\Conc{\species{7}}+2\Conc{\species{1}}\Conc{\species{8}}+2\Conc{\species{1}}\Conc{\species{10}}-\Conc{\species{1}}\Conc{\species{11}}+2\Conc{\species{2}}\Conc{\species{4}}\\
 & \quad +2\Conc{\species{2}}\Conc{\species{6}}+\Conc{\species{3}}\Conc{\species{4}}+\Conc{\species{3}}\Conc{\species{6}}+3\Conc{\species{3}}\Conc{\species{9}})-4\Conc{\species{11}}k^{eff}_{2,1}-\Conc{\species{11}}k^{eff}_{2,0}\\
 & \quad -2\Conc{\species{11}}k^{eff}_{1,1}-2\Conc{\species{11}}k^{eff}_{3,0}-\Conc{\species{11}}k^{eff}_{0,2}+6\Conc{\species{12}}k^{eff}_{2,1}-\delta\Conc{\species{11}}\\[0.5em]
\DCDtNoSpacing{\species{12}} &= \kon{}(\Conc{\species{1}}\Conc{\species{11}}+2\Conc{\species{2}}\Conc{\species{8}}+\Conc{\species{3}}\Conc{\species{10}}+\Conc{\species{4}}^2+\Conc{\species{6}}^2+3\Conc{\species{9}}^2)-6\Conc{\species{12}}k^{eff}_{2,1}\\
 & \quad -3\Conc{\species{12}}k^{eff}_{4,0}-6\Conc{\species{12}}k^{eff}_{2,2}-6\Conc{\species{12}}k^{eff}_{4,1}-\Conc{\species{12}}k^{eff}_{0,3}-\delta\Conc{\species{12}}\\[0.5em]
\end{align*}
%\end{split}
%\end{equation}
%\end{align*}
%\end{align*}

\end{document}
